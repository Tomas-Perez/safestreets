In this section, the requirements specified in chapter 3 of the RASD document are categorized according to their implementation state.\\

\subsection{Included requirements} \label{sub-sect:inc-req}
The following requirements were fully implemented:

\begin{itemize}[label={}]
            \item {[FR1]} - The user is able to take pictures from the mobile application to add to a report.
            \item {[FR2]} - The user is able to fill out a form providing information about a traffic violation, consisting of:
            \begin{itemize}[label={\textbullet}]
                \item Type of violation
                \item License plate
                \item At least one photo of the scene
            \end{itemize}
            \item {[FR3]} - The application can determine date, time and location based on information provided by the device when the pictures are taken and adds this metadata to the report.
            \item {[FR4]} - The user can submit a full report to the system

            \item {[FR5]} - The user is provided a map where the location of reports are indicated by markers.
            \item {[FR6]} - The user can filter the reports shown on the map by date and type of violation.

            \item {[FR9]} - A secured API is provided to obtain and filter report information. The following filters are possible:
            \begin{itemize}[label={\textbullet}]
                \item Date and time
                \item Location
                \item Type of violation
            \end{itemize}

            \item {[SR1]} - Only photos taken through the mobile application are submitted, the user cannot upload a picture previously saved on their device.

            \item {[SR2]} - Special API keys are generated and provided to authorities
            \item {[FR10]} - A secured API accessible only to authorities, provides the following information about reports:
            \begin{itemize}[label={\textbullet}]
                \item Photos
                \item License plate
            \end{itemize}
            
            \item {[FR13]} - The API includes the system’s degree of confidence in the report to the provided report information after a query.
            \item {[FR14]} - Report information provided by the API can be filtered by their degree of confidence.
            \item {[FR15]} - Users can review photos with a license plate recognition lower than 80\% to adjust the confidence in the recognition.

            \item {[FR18]} - The user can register by inputting his full name and email, and choosing a username and password.
            \item {[FR19]} - The user has access to view his full account information.
            \item {[FR20]} - The user can edit his email and full name.
            \item {[FR21]} - An API key is provided to the user via their profile screen in the mobile application
            \item {[FR22]} - The user can login to the mobile application by providing their email and password.
\end{itemize}


\subsection{Partially included requirements}

Location is shown, and searched, by coordinates only. This is because, in order to get an approximate address, it is required to integrate with the Google Places API, which is paid per request.
\begin{itemize}[label={}]

    \item {[FR7]} - The user can search for a specific location on the map by inputting coordinates or an address.
    \item {[FR8]} - The user can select a report on the map and obtain information about it. This information includes:
    \begin{itemize}[label={\textbullet}]
        \item Report ID: unique reference to each report
        \item Type of violation
        \item Date and time
        \item Location: gps coordinates, approximate street name and number
    \end{itemize}
\end{itemize}

\vspace{5mm}
Since there was no integration with an external license plate registry, the degree of confidence is based on the license plate recognition only.
\begin{itemize}[label={}]

            \item {[FR12]} - A degree of confidence is assigned to each report, where a high confidence report will have:
                \begin{itemize}[label={\textbullet}]
                    \item License plate recognition of 80\%+ confidence.
                    \item Car recognition of 80\%+ confidence.
                    \item The license plate and car pair are found in the license plate registry.
                \end{itemize}
            If any of these criteria are not met the report is considered low confidence.

\end{itemize}


\subsection{Excluded requirements}
As previously mentioned, currently there is no integration with an external license plate registry.
\begin{itemize}[label={}]
    \item {[FR11]} - Detected cars and license plates are cross referenced with the license plate registry to ensure the license plate belongs to the car.
\end{itemize}

The following requirements belong to the advanced functionality, which was not required for the implementation:

\begin{itemize}[label={}]
            \item {[SR3]} - End-to-end encryption is provided for the submission of reports. 
            \item {[FR16]} - Information obtained from the ticketing system is used to determine if a report contributed to the issuing of a traffic ticket.
            \item {[FR17]} - Whether the report contributed to a traffic ticket is included in the information provided by the mobile application and the API.
\end{itemize}