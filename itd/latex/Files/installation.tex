\subsection{Back-end server and database}
The recommended way of installing the application is through Docker and the provided docker-compose file. This will set up both the backend server and the mongodb database.\\

\textbf{Pre-requisites:}
\begin{itemize}[label={\textbf{-}}]
    \item 
    Have Docker installed in your computer, can be downloaded from: \url{https://www.docker.com/}
    \item 
    Have an internet connection for installation.
\end{itemize}

\textbf{Installation steps:}
\begin{itemize}[label={\textbf{-}}]
    \item 
    Make sure Docker is running.
    \item 
    Open the terminal/command line at the directory containing the provided docker-compose.yml file.
    \item 
    Type and run the following command: docker-compose up.
    \item 
    Wait for Docker to finish the setup.
    \item 
    To check if everything is ok, open a browser and navigate to http://localhost:8080/auth/ping. The text “pong” should be displayed on your screen.
\end{itemize}

\subsection{Mobile}
To install the mobile application:

\textbf{Pre-requisites:}
\begin{itemize}[label={\textbf{-}}]
    \item 
    Android phone.
    \item 
    Application server running.
\end{itemize}

\textbf{Installation steps:}
\begin{itemize}[label={\textbf{-}}]
    \item 
    Transfer the provided APK file to the phone.
    \item 
    On the phone, navigate to the directory containing the APK and tap on the file.
    \item 
    Some warnings may pop up, as the application is not registered, ignore them and continue with the installation.
    \item 
    The app is installed.
    \item 
    Open the app and do a tap and hold on the grey “SafeStreets” title.
    \item 
    An alert will pop up to enter the URL to connect to the application server.
    \begin{itemize}[label={\textbf{-}}]
        \item Note: instead of localhost, the URL will have the ip of the machine running the server, for example http://192.168.99.100:8080.
    \end{itemize}
    \item Press connect.
    \item A tick mark should appear next to the URL field, indicating a good connection, and the “SafeStreets” title will change to blue.
    \item You can either sign up as a new user or use a preexisting user, for example: 
        \begin{itemize}[label={\textbf{-}}]
            \item email: ewan@mail.com
            \item password: password
        \end{itemize}
\end{itemize}