\subsection{Purpose}

The purpose of this document is to provide a description of the SafeStreets system. A detailed explanation of the proposed solution is given, along with the requirements and assumptions made to achieve it.

SafeStreets is a crowd-sourced application that	 intends to provide users with the possibility to notify authorities when traffic violations occur, and in particular	 parking violations. With the amount of traffic we are seeing nowadays, it is hard to maintain order throughout the entire city, so the help of the community is more than welcome.
The application allows users to report violations by sending pictures, along with important information, like the date, time and position.
Examples of violations are vehicles parked in the middle of bike lanes or in places reserved for people with disabilities, double parking, and so on.
The system also allows both end users and authorities to access the information gathered, with different levels of visibility depending on the roles.
With the information provided, it is then possible for the municipality to integrate it with their traffic ticket system and automatically issue the corresponding ticket to a reported offender. This will accelerate the whole process, saving time and money to the state and could eventually result in a decrease in violations.
At the same time, the ticketing system can provide information to SafeStreets, which presents the possibility of building statistics such as the most egregious offenders and analyse the effect of the application by looking at the trend in violations.

\subsection{Scope}

As already mentioned, the SafeStreets system is designed to provide users with the ability to report traffic violations through an application. In order to achieve this, the user needs to take and upload photos of the scene, which captures the license plate of the vehicle committing the violation and complete a form with information such as the license plate number and the type of violation. This data, along with metadata retrieved from the user’s device (geographical position, date and time) is then sent to the system.

\subsubsection{Goals}
\begin{itemize}
\item
[G1] - The user is able to report a traffic violation to authorities.
\item
[G2] - License plates can be recognized from the pictures of the violation report.
\item
[G3] - Roles with different levels of permission are assigned to users and authorities.
\item
[G4] - The information gathered from the reports is provided to users and authorities according to their role.
\item
[G5] - The system must protect the chain of custody of the reports. 
\item
[G6] - Reports that had their integrity compromised and malicious reports will be detected and discarded.
\item
[G7] - Information about issued tickets provided by the municipality system can be cross referenced with the SafeStreets database and analysed.
\end{itemize}

\subsubsection{Definitions, Acronyms, Abbreviations}
\subsubsection{Revision history}
\subsubsection{Reference documents}
\subsubsection{Document structure}