\subsection{Purpose}

The purpose of this document is to provide a description of the SafeStreets system. A detailed explanation of the proposed solution is given, along with the requirements and assumptions made to achieve it.

SafeStreets is a crowd-sourced application that	 intends to provide users with the possibility to notify authorities when traffic violations occur, and in particular	 parking violations. With the amount of traffic we are seeing nowadays, it is hard to maintain order throughout the entire city, so the help of the community is more than welcome.

The application allows users to report violations by sending pictures, along with important information, like the date, time and position.

Examples of violations are vehicles parked in the middle of bike lanes or in places reserved for people with disabilities, double parking, and so on.

The system also allows both end users and authorities to access the information gathered, with different levels of visibility depending on the roles.

With the information provided, it is then possible for the municipality to integrate it with their traffic ticket system and automatically issue the corresponding ticket to a reported offender. This will accelerate the whole process, saving time and money to the state and could eventually result in a decrease in violations.

At the same time, the ticketing system can provide information to SafeStreets, which presents the possibility of building statistics such as the most egregious offenders and analyse the effect of the application by looking at the trend in violations.

\subsection{Scope}

\subsubsection{General purpose}

As already mentioned, the SafeStreets system is designed to provide users with the ability to report and get information of reported traffic violations through an application.

Any user with a device capable of running the application can sign up to the system, which enables them to access its functionalities.

In order to submit a report, the user needs to fill a form. In it they have to enter the license plate number of the vehicle committing the violation, the type of violation and at least one photo of the scene, where the license plate of the vehicle can be easily recognized. This data, along with metadata retrieved from the user's device (geographical position, date and time) is then sent to the system.

The system is responsible for analysing the validity of the report. To achieve this, a license plate recognition algorithm is utilized. The output consists of possibly multiple license plates (the picture could include more than one car), along with the certainty of the detection.

If multiple license plates are detected, the target of the report is determined by comparing it with the input of the user in the form.

After a target license plate is confirmed, the confidence of the detection is evaluated, if it is below a certain threshold, it must pass through a community review. During this process, multiple users willing to participate are shown a cutout of the license plate in the picture and asked to input what they see. If a consensus is reached, then the report is considered valid.

The data collected by the system in relation to reports is to be queried by its users. There are two distinct targets of this functionality: standard users and the municipality. The main difference between the two is that the municipality can access information that should not be freely accessible to everyone because of security and privacy concerns. Through the application, users are capable of visualizing a city map showing where the violations happened. Furthermore, a public API is made available, facilitating data analysis and system integration.

\subsubsection{World and Machine phenomena}
To mark the boundaries of the system, here we denote:
    \begin{itemize}
        \item The world phenomena which concern the system (the machine).
        \item The phenomena internal to the machine from a high level point of view.
        \item Shared phenomena that cross from the world to the machine or vice versa.
    \end{itemize}

\begin{table}[H]
    \centering
    \begin{tabular}{|p{10cm}|c|c|}
    \hline
    \textbf{Phenomenon} & \textbf{Shared} & \textbf{Controlled by} \\ \hline
    A person commits a traffic violation & N & World \\ \hline
    User spots a traffic violation & N & World \\ \hline
    User logs in & Y & World \\ \hline
    User fills a report form & Y & World \\ \hline
    User takes pictures of the traffic violation & Y & World \\ \hline
    User submits report & Y & World \\ \hline
    Machine analyzes the pictures to find a license plate & N & Machine \\ \hline
    Machine accepts the submitted report & Y & Machine \\ \hline
    Machine rejects the submitted report & Y & Machine \\ \hline
    User wants to find information about traffic violations & N & World \\ \hline
    User requests report information in a specific area and time & Y & World \\ \hline
    Machine receives the request and filters its stored reports according to the query & N & Machine \\ \hline
    Machine answers the request for information & Y & Machine \\ \hline
    Authority uses report information to generate traffic tickets & N & World \\ \hline
    Machine requests information about traffic tickets to an external traffic ticket system & Y & Machine \\ \hline
    Traffic ticket system responds to the information request & Y & World \\ \hline
    Machine cross references its stored report information with the traffic ticket system response and analyses it & N & Machine \\ \hline
    \end{tabular}
    \caption{\label{tbl} World and Machine phenomena}
    \end{table}

\subsubsection{Goals}
\begin{itemize}
\item
[G1] - The user is able to report a traffic violation to authorities.
\item
[G2] - The user is able to visualize reports in a specified area and time.
\item
[G3] - It is possible to query report information in an easily parsable format to allow for data analysis.
\item
[G4] - Only authorities are able to access report pictures and license plates.
\item
[G5] - The system must protect the chain of custody of the reports. 
\item
[G6] - Compromised reports are detected and discarded.
\item
[G7] - Information about issued tickets provided by the municipality system can be cross referenced with the SafeStreets database and analysed.
\end{itemize}

\subsection{Definitions, Acronyms, Abbreviations}
\subsubsection{Definitions}
\begin{itemize}
    \item Compromised report: a report that has been modified by an unauthorized agent outside the system boundaries.
    \item Authority: a local agency whose purpose is, as indicated by the current law, to enforce traffic rules. For example: the police.      
\end{itemize}

\subsubsection{Revision history}
\begin{itemize}
    \item Version 1.0: First release.
\end{itemize}

\subsubsection{Reference documents}

\subsubsection{Document structure}
The RASD document is composed of five chapters:\\\\
Chapter 1: The problem is introduced. A description of the purpose of the application is given, followed by the scope, where the world and machine phenomena are explained, along with the system goals. Also, definitions are listed to help the reader understand the concepts used.\\\\
Chapter 2: An overall description of the product. Including a further detailed description of the product, with the help of class and state diagrams, a description of the main functionalities, the different types of actors that interact with the system, and the domain assumptions considered for solving the problem.\\\\
Chapter 3: Specific requirements. This is the main chapter of the document. It includes the external interface requirements, like the user and software interfaces. Scenarios for typical application usage are provided, followed by functional and performance requirements, and the constraints under which the system needs to function. Lastly, the system software attributes are discussed.\\\\
Chapter 4: In this chapter a documented Alloy model is presented for a formal analysis of the problem, along with a discussion of the purpose of this model and what it proves.\\\\
Chapter 5: Shows the effort spent by each member of the group in the development of the document.
